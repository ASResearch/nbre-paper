% !TEX root = main.tex

\section{系统实现}

本章详细说明中间表示模块和即时编译引擎的实现细节,包括其中的设计思想以及算法的具体实现。
对于系统中其它模块的具体实现,并不是文章的重点,就不再详细说明。

\subsection{中间表示管理模块}

中间表示管理模块接收协议表示集合,插入并发执行代码到协议表示集合中。
其中实施插入并发执行代码的过程具体分为以下几个阶段:
\begin{itemize}
  \item 初始化阶段。
  通过协议表示集合的$.data$字段获取协议表示集合中的共享变量。
  对每一个共享变量都维护一个数据结构,包括共享变量的版本号以及写锁,其中版本号初始化为$0$,写锁初始化为打开。
  对共享变量所在的协议表示,维护一对读写集,以日志事件的形式记录对共享变量的读写操作,读集和写集初始化都为空。
  维护一个全局版本的时钟,对每一次执行操作,都有一个当前执行的读版本,当前执行的读版本初始化为全局版本时钟。

  \item 执行阶段。
  对于读集,会将读取共享变量地址加入到读集的条目中;
  对于写集,会将写入共享变量地址和值加入到写集中。
  读操作会先访问写集,若写集中存在要访问的共享变量地址,读操作直接将写集中对应共享变量地址对应地取出,否则读操作直接访问共享变量;
  除此之外,需要检查该共享变量地址版本是否小于等于当前执行版本,以及共享变量是否上锁,若不满足条件则说明共享变量正在被修改或已经被修改,需要立即中止执行;
  写操作会直接将共享变量地址及与之对应的数值写入写集,并检查共享变量版本号及锁,不满足立即中止执行。

  \item 提交阶段。
  对全局版本锁进行自增操作,并设置当前执行的写版本为自增后的全局版本锁。
  对读集提交,检查共享内存版本号及是否上锁。
  对写集提交,对每个共享变量上锁,将写集的日志更新到变量,并将共享变量版本更新为当前执行的写版本,解锁共享变量。

\end{itemize}

\subsection{即时编译引擎}
