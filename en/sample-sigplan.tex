%%
%% This is file `sample-sigplan.tex',
%% generated with the docstrip utility.
%%
%% The original source files were:
%%
%% samples.dtx  (with options: `sigplan')
%%
%% IMPORTANT NOTICE:
%%
%% For the copyright see the source file.
%%
%% Any modified versions of this file must be renamed
%% with new filenames distinct from sample-sigplan.tex.
%%
%% For distribution of the original source see the terms
%% for copying and modification in the file samples.dtx.
%%
%% This generated file may be distributed as long as the
%% original source files, as listed above, are part of the
%% same distribution. (The sources need not necessarily be
%% in the same archive or directory.)
%%
%% The first command in your LaTeX source must be the \documentclass command.
\documentclass[sigplan,screen]{acmart}

%%
%% \BibTeX command to typeset BibTeX logo in the docs
\AtBeginDocument{%
  \providecommand\BibTeX{{%
    \normalfont B\kern-0.5em{\scshape i\kern-0.25em b}\kern-0.8em\TeX}}}

%% Rights management information.  This information is sent to you
%% when you complete the rights form.  These commands have SAMPLE
%% values in them; it is your responsibility as an author to replace
%% the commands and values with those provided to you when you
%% complete the rights form.
\setcopyright{acmcopyright}
\copyrightyear{2018}
\acmYear{2018}
\acmDOI{10.1145/1122445.1122456}

%% These commands are for a PROCEEDINGS abstract or paper.
\acmConference[Woodstock '18]{Woodstock '18: ACM Symposium on Neural
  Gaze Detection}{June 03--05, 2018}{Woodstock, NY}
\acmBooktitle{Woodstock '18: ACM Symposium on Neural Gaze Detection,
  June 03--05, 2018, Woodstock, NY}
\acmPrice{15.00}
\acmISBN{978-1-4503-9999-9/18/06}


%%
%% Submission ID.
%% Use this when submitting an article to a sponsored event. You'll
%% receive a unique submission ID from the organizers
%% of the event, and this ID should be used as the parameter to this command.
%%\acmSubmissionID{123-A56-BU3}

%%
%% The majority of ACM publications use numbered citations and
%% references.  The command \citestyle{authoryear} switches to the
%% "author year" style.
%%
%% If you are preparing content for an event
%% sponsored by ACM SIGGRAPH, you must use the "author year" style of
%% citations and references.
%% Uncommenting
%% the next command will enable that style.
%%\citestyle{acmauthoryear}

%%
%% end of the preamble, start of the body of the document source.
\usepackage{tikz}
\usepackage{pgfplots}
\usetikzlibrary{positioning}
\usetikzlibrary{fit}
\usetikzlibrary{snakes}
\usetikzlibrary{shapes.geometric}
\usetikzlibrary{patterns}
\usetikzlibrary{shapes,arrows,chains}
\usetikzlibrary{calc}
\usetikzlibrary{positioning, fit}
\usetikzlibrary{backgrounds}
\usetikzlibrary{intersections}

\begin{document}

%%
%% The "title" command has an optional parameter,
%% allowing the author to define a "short title" to be used in page headers.
\title{Blockchain Runtime Environment: An Accompanying System for Upgrading Blockchain}

%%
%% The "author" command and its associated commands are used to define
%% the authors and their affiliations.
%% Of note is the shared affiliation of the first two authors, and the
%% "authornote" and "authornotemark" commands
%% used to denote shared contribution to the research.
%\author{Ben Trovato}
%\authornote{Both authors contributed equally to this research.}
%\email{trovato@corporation.com}
%\orcid{1234-5678-9012}
%\author{G.K.M. Tobin}
%\authornotemark[1]
%\email{webmaster@marysville-ohio.com}
%\affiliation{%
  %\institution{Institute for Clarity in Documentation}
  %\streetaddress{P.O. Box 1212}
  %\city{Dublin}
  %\state{Ohio}
  %\postcode{43017-6221}
%}

%\author{Lars Th{\o}rv{\"a}ld}
%\affiliation{%
  %\institution{The Th{\o}rv{\"a}ld Group}
  %\streetaddress{1 Th{\o}rv{\"a}ld Circle}
  %\city{Hekla}
  %\country{Iceland}}
%\email{larst@affiliation.org}

%\author{Valerie B\'eranger}
%\affiliation{%
  %\institution{Inria Paris-Rocquencourt}
  %\city{Rocquencourt}
  %\country{France}
%}

%\author{Aparna Patel}
%\affiliation{%
 %\institution{Rajiv Gandhi University}
 %\streetaddress{Rono-Hills}
 %\city{Doimukh}
 %\state{Arunachal Pradesh}
 %\country{India}}

%\author{Huifen Chan}
%\affiliation{%
  %\institution{Tsinghua University}
  %\streetaddress{30 Shuangqing Rd}
  %\city{Haidian Qu}
  %\state{Beijing Shi}
  %\country{China}}

%\author{Charles Palmer}
%\affiliation{%
  %\institution{Palmer Research Laboratories}
  %\streetaddress{8600 Datapoint Drive}
  %\city{San Antonio}
  %\state{Texas}
  %\postcode{78229}}
%\email{cpalmer@prl.com}

%\author{John Smith}
%\affiliation{\institution{The Th{\o}rv{\"a}ld Group}}
%\email{jsmith@affiliation.org}

%\author{Julius P. Kumquat}
%\affiliation{\institution{The Kumquat Consortium}}
%\email{jpkumquat@consortium.net}

%%
%% By default, the full list of authors will be used in the page
%% headers. Often, this list is too long, and will overlap
%% other information printed in the page headers. This command allows
%% the author to define a more concise list
%% of authors' names for this purpose.
\renewcommand{\shortauthors}{Trovato and Tobin, et al.}

%%
%% The abstract is a short summary of the work to be presented in the
%% article.
\begin{abstract}
\end{abstract}

%%
%% The code below is generated by the tool at http://dl.acm.org/ccs.cfm.
%% Please copy and paste the code instead of the example below.
%%
\begin{CCSXML}
<ccs2012>
 <concept>
  <concept_id>10010520.10010553.10010562</concept_id>
  <concept_desc>Computer systems organization~Embedded systems</concept_desc>
  <concept_significance>500</concept_significance>
 </concept>
 <concept>
  <concept_id>10010520.10010575.10010755</concept_id>
  <concept_desc>Computer systems organization~Redundancy</concept_desc>
  <concept_significance>300</concept_significance>
 </concept>
 <concept>
  <concept_id>10010520.10010553.10010554</concept_id>
  <concept_desc>Computer systems organization~Robotics</concept_desc>
  <concept_significance>100</concept_significance>
 </concept>
 <concept>
  <concept_id>10003033.10003083.10003095</concept_id>
  <concept_desc>Networks~Network reliability</concept_desc>
  <concept_significance>100</concept_significance>
 </concept>
</ccs2012>
\end{CCSXML}

\ccsdesc[500]{Computer systems organization~Embedded systems}
\ccsdesc[300]{Computer systems organization~Redundancy}
\ccsdesc{Computer systems organization~Robotics}
\ccsdesc[100]{Networks~Network reliability}

%%
%% Keywords. The author(s) should pick words that accurately describe
%% the work being presented. Separate the keywords with commas.
\keywords{}

%% A "teaser" image appears between the author and affiliation
%% information and the body of the document, and typically spans the
%% page.
%\begin{teaserfigure}
  %\includegraphics[width=\textwidth]{sampleteaser}
  %\caption{Seattle Mariners at Spring Training, 2010.}
  %\Description{Enjoying the baseball game from the third-base
  %seats. Ichiro Suzuki preparing to bat.}
  %\label{fig:teaser}
%\end{teaserfigure}

%%
%% This command processes the author and affiliation and title
%% information and builds the first part of the formatted document.
\maketitle

\section{Introduction}

\section{Background}

\section{Architecture}
In this section, we first define blockchain protocol and version, and illustrate
blockchain abstract description based on the definition. We then explain the
architecture designing purpose and principle. After that, we introduce Blockchain
Runtime Environment architecture, system components and working flow included.

\subsection{Overview}
Blockchain Runtime Environment (BRE) is an accompanying system for upgrading,
both permissionless and permissioned, blockchain. As the name says, BRE
provides the environment for blockchain running. Generally, BRE is not going to
run independently, but will interact with blockchain when manipulating
updating. Specifically, for a blockchain node, which has installed BRE, is
running a blockchain client (or simply client) process and a friend process BRE
at the same time.

Upgrading blockchain, actually, is upgrading blockchain protocol.
Blockchain protocol is the parameters that blockchain holds, including block
generation rate, block size, transaction fees and et al. To be
more broadly, any line of code in blockchain system is blockchain protocol.
Blockchain upgrading happens when blockchain system meets serious problem,
which results in bug fix in an emergency; or when new features launches,
leading to functions developing correspondingly. Both of these situations
are blockchain upgrading, and moreover, introducing blockchain system code
upgrading (optimizing and revising), which explains blockchain protocol is
blockchain system code itself.

However, it should be noted that the application layer (based on blockchain)
upgrading is not blockchain upgrading. Smart contract upgrading for
example, is outside blockchain upgrading scope.

Blockchain protocol version, which also represents as blockchain system
version, increases after blockchain protocol enforcing upgrade. For traditional
software upgrade, only the lastest version of the system needs to be
maintained, and backward compatible is made when the latest is
released. It barely cares about the historical versions of the system. But in
contrast, blockchain system needs to retain all historical protocols and its
versions. If not, transactions validation may fail when the absent of specific
version of protocol, because data consistency is strictly required for
blockchain system.

In view of above-mentioned description, blockchain node, blockchain protocol
and blockchain node state is as follows:
\begin{itemize}
  \item $n_i$ is blockchain node with identity $i=1,2,...$;
  \item $p_j$ is blockchain protocol with version $j=1,2,...$;
  \item $S_{m}^{j}$ is blockchain node state, where the node holds a set of $m$
    kinds of historical protocols $\{p_1,p_2,..,p_j,..,p_m\}$, for
    $1{\leq}j{\leq}m$, and the node is running with protocol $p_j$ currently.
\end{itemize}

Thus, we further abstract blockchain system along with node state.
For any node $n_i$ of blockchain, at moment $t$,
the node state is $S_{m}^{j}$, which means the node holds the set of protocols
$\{p_1,p_2,..,p_j,..,p_m\}$, and now the node's running protocol is $p_j$.
Obviously, blockchain nodes' state may differ from each other: the set of
protocols are originally different or the current running protocols ars aree
different. By default (not receiving requests from client),
blockchain nodes' state is always $S_{m}^{m}$, namely each node has entire
historical protocols and runs the lastest protocol, only if last irreversible
block (LIB) has been validated.

\begin{figure}
  \includegraphics[width=\linewidth]{pdfs/fig1.pdf}
  \caption{Blockchain system abstraction}
\end{figure}

As illustrating in Fig. 1-a), suppose that there are at most three protocols
(set $\{p_1,p_2,p_3\}$) for the having-LIB node. Therefore,
blockchain node $n_1$, $n_2$, $n_5$, $n_7$ are in the state of default
(state $S_3^3$), while not for the others. Several reasons result in node not
running in default state: 1) Node is in the stage of block synchronization,
executing one historical protocol for block validation, such as node $n_3$ and
$n_4$; 2) Node received external request, and is running a specific assigned
protocol, like $n_6$ and $n_8$.

As soon as new protocol $p_4$ is deployed to blockchain system,
$p_4$ will be broadcast to all nodes. After all, nodes with having-LIB
will receive $p_4$. Finally, nodes will add $p_4$ to their protocol set, start
running $p_4$, and update their default state to $S_4^4$, as Fig. 1-b) says.

Most of the clients do not aware of blockchain's upgrading unless the client
gets BRE response that can identify upgrading event after requesting. Thus the
hallmark of successful upgrading is that the response that the client gets is
exactly the protocol running result supposed to be.

Three more need to take into concern when designing and implementing its
architecture:
\begin{itemize}
  \item Protocol deployment. Base on the description of protocol in the narrow
    and the general sense, protocol, with the underlying libraries of
    blockchain system, implements some requirements or functions; Furthermore,
    protocol as a part of the underlying libraries, provides blockchain services.
    In a word, protocol actually is blockchain system code. On the other hand,
    protocol needs to be deployed to the blockchain system and broadcast to
    other nodes. Fortunately, blockchain's sending transaction property meets
    the need of protocol deployment and broadcasting. So, consider of packing
    protocol (source code) into transaction payload, and deploy it by just
    sending a special transaction.
  \item Protocol management. Previously, it has stated the fact that blockchain
    upgrading produces various versions of protocol, resulting in the
    requirement of protocol version control. Managering newly comming protocol
    and responsing the client's request for specific protocol executing result
    are the main parts of protocol management. For newly comming protocol,
    which is packed into the transaction payload, it needs to be parsed from
    block tranaction and collected, and then saved persistently. For responsing
    the request, it needs to retrive the protocol that matches the query
    condition efficiently for further execution.
  \item Protocol execution. Generally, protocol execution proceeds right after external
    request arrives. As mentioned above, protocol actually is source code, and
    source code representation executable if it can be executed directly as in
    machine code, or indirectly using a interpreter. Unfortunately, protocol
    entity, coming from transaction payload parse and collection, is not
    representation executable: protocol is high-level language, and a
    high-level language finally become executable until compilation procedure
    (preprocess, compile, assemble and link) done; The same to its
    dependencies. But if manipulating compilation procedure each time when
    request arrives seems impractical, which will affect the efficiency of
    execution, lead to disastrous results for performance.

    Just-In-Time technology is the answer of protocol-execution problem. If
    protocol entity is not source code anymore, but a translated language
    representation called protocol representation that could be executed
    indirectly by JIT interpreter. There will be no compilation procedure when
    executing protocol representation but check variables and functions in
    symbol table and interprete dynamically. Thus involve translation from
    protocol to protocol representation when parsing transaction payload for
    real protocol entity.
\end{itemize}

Blockchain Runtime Environment architecture consists of components (see in Fig.
2) including:
\begin{itemize}
  \item \textit{Protocol representation manager}
  \item \textit{Protocol representation database}
  \item \textit{Compiler frontend}
  \item \textit{Intermediate representation module manager}
  \item \textit{Just-In-Time engine}
\end{itemize}

\begin{figure}
  \includegraphics[width=\linewidth,trim={0 0 .8cm 0},clip]{pdfs/fig2.pdf}
  \caption{Blockchain Runtime Environment architecture}
\end{figure}

The function of these components along with protocol deployment, upgrade and
execution procedure will be illustrated in the next three sections.

\subsection{Protocol Deployment}

\begin{figure}
  \includegraphics[width=\linewidth,trim={0 0 2.2cm 0},clip]{pdfs/fig3.pdf}
  \caption{Protocol deployment in transaction payload}
\end{figure}

Fig. 3 shows that blockchain system in a logical-level way. Blocks are linked
one by another, because of child block holding its parent block hash. In a
block, transactions are deterministic in specific order. For each transaction,
it has transaction meta data (source, target, value, type, et al.) and
transaction payload. Usually, transaction payload is empty for transfer
transaction, but argument or smart contract content for smart contract call
or deployment transaction respectively.
The protocol for blockchain upgrading comes from transaction payload
(Sec. 3.1). Specifically, protocol, actually high-level source
code, and its meta data (protocol name, protocol version and its dependencies'
meta data) will be organized and packed into transaction payload field.
There will be a new type of transaction named \textit{protocol} since no such
transaction type for upgrading.
Only one version of protocol is allowed in one transaction, but not limit the
number of protocols in one block. Finally, the committee signatures
\textit{protocol} transaction and broadcasts it.

\textit{Discussion on design choices.} The way of protocol deployment takes use
of blockchain transacton's feature: carrying information and spreading to all
nodes. Deploying protocol by sending transaction with protocol in it instead of
active downloading from centralized server seems more reasonable in blockchain
world. On the other hand, unlike traditional distributed system or permissioned
blockchain, who have an identity, nodes are free to join in or leave in
permissionless blockchain, thus makes it impossible to determine the set of
upgrading nodes in advance.

It's very similar to smart contract
deployment, call and execution phase when compared with protocol upgrading
procedure. Is it possible to manipulate blockchain upgrading by smart contract
regardless of protocol? The answer is definitely yes, in some conditions. Smart
contract is limited by on-chain data interaction, time execution and authority.
Smart contract has access to on-chain data only if corresponding interface has
been defined, while protocol is free to invoke any functions of BRE. Besides,


\subsection{Protocol Upgrade}
\textit{Discussion on design choices.}

\subsection{Protocol Execution}
\textit{Discussion on design choices.}

%\section{Tables}

%The ``\verb|acmart|'' document class includes the ``\verb|booktabs|''
%package --- \url{https://ctan.org/pkg/booktabs} --- for preparing
%high-quality tables.

%Table captions are placed {\itshape above} the table.

%Because tables cannot be split across pages, the best placement for
%them is typically the top of the page nearest their initial cite.  To
%ensure this proper ``floating'' placement of tables, use the
%environment \textbf{table} to enclose the table's contents and the
%table caption.  The contents of the table itself must go in the
%\textbf{tabular} environment, to be aligned properly in rows and
%columns, with the desired horizontal and vertical rules.  Again,
%detailed instructions on \textbf{tabular} material are found in the
%\textit{\LaTeX\ User's Guide}.

%Immediately following this sentence is the point at which
%Table~\ref{tab:freq} is included in the input file; compare the
%placement of the table here with the table in the printed output of
%this document.

%To set a wider table, which takes up the whole width of the page's
%live area, use the environment \textbf{table*} to enclose the table's
%contents and the table caption.  As with a single-column table, this
%wide table will ``float'' to a location deemed more
%desirable. Immediately following this sentence is the point at which
%Table~\ref{tab:commands} is included in the input file; again, it is
%instructive to compare the placement of the table here with the table
%in the printed output of this document.

%\begin{table*}
  %\caption{Some Typical Commands}
  %\label{tab:commands}
  %\begin{tabular}{ccl}
    %\toprule
    %Command &A Number & Comments\\
    %\midrule
    %\texttt{{\char'134}author} & 100& Author \\
    %\texttt{{\char'134}table}& 300 & For tables\\
    %\texttt{{\char'134}table*}& 400& For wider tables\\
    %\bottomrule
  %\end{tabular}
%\end{table*}

%\begin{table}
  %\caption{Frequency of Special Characters}
  %\label{tab:freq}
  %\begin{tabular}{ccl}
    %\toprule
    %Non-English or Math&Frequency&Comments\\
    %\midrule
    %\O & 1 in 1,000& For Swedish names\\
    %$\pi$ & 1 in 5& Common in math\\
    %\$ & 4 in 5 & Used in business\\
    %$\Psi^2_1$ & 1 in 40,000& Unexplained usage\\
  %\bottomrule
%\end{tabular}
%\end{table}

%\section{Math Equations}
%You may want to display math equations in three distinct styles:
%inline, numbered or non-numbered display.  Each of the three are
%discussed in the next sections.

%\subsection{Inline (In-text) Equations}
%A formula that appears in the running text is called an inline or
%in-text formula.  It is produced by the \textbf{math} environment,
%which can be invoked with the usual
%\texttt{{\char'134}begin\,\ldots{\char'134}end} construction or with
%the short form \texttt{\$\,\ldots\$}. You can use any of the symbols
%and structures, from $\alpha$ to $\omega$, available in
%\LaTeX~\cite{Lamport:LaTeX}; this section will simply show a few
%examples of in-text equations in context. Notice how this equation:
%\begin{math}
  %\lim_{n\rightarrow \infty}x=0
%\end{math},
%set here in in-line math style, looks slightly different when
%set in display style.  (See next section).

%\subsection{Display Equations}
%A numbered display equation---one set off by vertical space from the
%text and centered horizontally---is produced by the \textbf{equation}
%environment. An unnumbered display equation is produced by the
%\textbf{displaymath} environment.

%Again, in either environment, you can use any of the symbols and
%structures available in \LaTeX\@; this section will just give a couple
%of examples of display equations in context.  First, consider the
%equation, shown as an inline equation above:
%\begin{equation}
  %\lim_{n\rightarrow \infty}x=0
%\end{equation}
%Notice how it is formatted somewhat differently in
%the \textbf{displaymath}
%environment.  Now, we'll enter an unnumbered equation:
%\begin{displaymath}
  %\sum_{i=0}^{\infty} x + 1
%\end{displaymath}
%and follow it with another numbered equation:
%\begin{equation}
  %\sum_{i=0}^{\infty}x_i=\int_{0}^{\pi+2} f
%\end{equation}
%just to demonstrate \LaTeX's able handling of numbering.

%\section{Citations and Bibliographies}

%The use of \BibTeX\ for the preparation and formatting of one's
%references is strongly recommended. Authors' names should be complete
%--- use full first names (``Donald E. Knuth'') not initials
%(``D. E. Knuth'') --- and the salient identifying features of a
%reference should be included: title, year, volume, number, pages,
%article DOI, etc.

%The bibliography is included in your source document with these two
%commands, placed just before the \verb|\end{document}| command:
%\begin{verbatim}
  %\bibliographystyle{ACM-Reference-Format}
  %\bibliography{bibfile}
%\end{verbatim}
%where ``\verb|bibfile|'' is the name, without the ``\verb|.bib|''
%suffix, of the \BibTeX\ file.

%Citations and references are numbered by default. A small number of
%ACM publications have citations and references formatted in the
%``author year'' style; for these exceptions, please include this
%command in the {\bfseries preamble} (before
%``\verb|\begin{document}|'') of your \LaTeX\ source:
%\begin{verbatim}
  %\citestyle{acmauthoryear}
%\end{verbatim}

  %Some examples.  A paginated journal article \cite{Abril07}, an
  %enumerated journal article \cite{Cohen07}, a reference to an entire
  %issue \cite{JCohen96}, a monograph (whole book) \cite{Kosiur01}, a
  %monograph/whole book in a series (see 2a in spec. document)
  %\cite{Harel79}, a divisible-book such as an anthology or compilation
  %\cite{Editor00} followed by the same example, however we only output
  %the series if the volume number is given \cite{Editor00a} (so
  %Editor00a's series should NOT be present since it has no vol. no.),
  %a chapter in a divisible book \cite{Spector90}, a chapter in a
  %divisible book in a series \cite{Douglass98}, a multi-volume work as
  %book \cite{Knuth97}, an article in a proceedings (of a conference,
  %symposium, workshop for example) (paginated proceedings article)
  %\cite{Andler79}, a proceedings article with all possible elements
  %\cite{Smith10}, an example of an enumerated proceedings article
  %\cite{VanGundy07}, an informally published work \cite{Harel78}, a
  %doctoral dissertation \cite{Clarkson85}, a master's thesis:
  %\cite{anisi03}, an online document / world wide web resource
  %\cite{Thornburg01, Ablamowicz07, Poker06}, a video game (Case 1)
  %\cite{Obama08} and (Case 2) \cite{Novak03} and \cite{Lee05} and
  %(Case 3) a patent \cite{JoeScientist001}, work accepted for
  %publication \cite{rous08}, 'YYYYb'-test for prolific author
  %\cite{SaeediMEJ10} and \cite{SaeediJETC10}. Other cites might
  %contain 'duplicate' DOI and URLs (some SIAM articles)
  %\cite{Kirschmer:2010:AEI:1958016.1958018}. Boris / Barbara Beeton:
  %multi-volume works as books \cite{MR781536} and \cite{MR781537}. A
  %couple of citations with DOIs:
  %\cite{2004:ITE:1009386.1010128,Kirschmer:2010:AEI:1958016.1958018}. Online
  %citations: \cite{TUGInstmem, Thornburg01, CTANacmart}. Artifacts:
  %\cite{R} and \cite{UMassCitations}.

%\section{Acknowledgments}

%Identification of funding sources and other support, and thanks to
%individuals and groups that assisted in the research and the
%preparation of the work should be included in an acknowledgment
%section, which is placed just before the reference section in your
%document.

%This section has a special environment:
%\begin{verbatim}
  %\begin{acks}
  %...
  %\end{acks}
%\end{verbatim}
%so that the information contained therein can be more easily collected
%during the article metadata extraction phase, and to ensure
%consistency in the spelling of the section heading.

%Authors should not prepare this section as a numbered or unnumbered {\verb|\section|}; please use the ``{\verb|acks|}'' environment.

%\section{Appendices}

%If your work needs an appendix, add it before the
%``\verb|\end{document}|'' command at the conclusion of your source
%document.

%Start the appendix with the ``\verb|appendix|'' command:
%\begin{verbatim}
  %\appendix
%\end{verbatim}
%and note that in the appendix, sections are lettered, not
%numbered. This document has two appendices, demonstrating the section
%and subsection identification method.

%%%
%%% The acknowledgments section is defined using the "acks" environment
%%% (and NOT an unnumbered section). This ensures the proper
%%% identification of the section in the article metadata, and the
%%% consistent spelling of the heading.
%\begin{acks}
%To Robert, for the bagels and explaining CMYK and color spaces.
%\end{acks}

%%%
%%% The next two lines define the bibliography style to be used, and
%%% the bibliography file.
%\bibliographystyle{ACM-Reference-Format}
%\bibliography{sample-base}

%%%
%%% If your work has an appendix, this is the place to put it.
%\appendix

%\section{Research Methods}

%\subsection{Part One}

%Lorem ipsum dolor sit amet, consectetur adipiscing elit. Morbi
%malesuada, quam in pulvinar varius, metus nunc fermentum urna, id
%sollicitudin purus odio sit amet enim. Aliquam ullamcorper eu ipsum
%vel mollis. Curabitur quis dictum nisl. Phasellus vel semper risus, et
%lacinia dolor. Integer ultricies commodo sem nec semper.

%\subsection{Part Two}

%Etiam commodo feugiat nisl pulvinar pellentesque. Etiam auctor sodales
%ligula, non varius nibh pulvinar semper. Suspendisse nec lectus non
%ipsum convallis congue hendrerit vitae sapien. Donec at laoreet
%eros. Vivamus non purus placerat, scelerisque diam eu, cursus
%ante. Etiam aliquam tortor auctor efficitur mattis.

%\section{Online Resources}

%Nam id fermentum dui. Suspendisse sagittis tortor a nulla mollis, in
%pulvinar ex pretium. Sed interdum orci quis metus euismod, et sagittis
%enim maximus. Vestibulum gravida massa ut felis suscipit
%congue. Quisque mattis elit a risus ultrices commodo venenatis eget
%dui. Etiam sagittis eleifend elementum.

%Nam interdum magna at lectus dignissim, ac dignissim lorem
%rhoncus. Maecenas eu arcu ac neque placerat aliquam. Nunc pulvinar
%massa et mattis lacinia.

\end{document}
\endinput
%%
%% End of file `sample-sigplan.tex'.
