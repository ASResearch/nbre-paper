% !TEX root = main.tex

\section{概要}

区块链系统(具体指公链)由于其天然的去中心化特性,其行为需要达成共识,升级并非易事。
区块链系统中没有一个可信机构可以发布安装更新,也没有一个相对中心化的升级节点提供升级系统的下载服务。
更进一步,区块链系统中客户端的行为是任意的,每一个客户端可以选择升级或者保持原有的状态,无法强制客户端升级。

区块链系统升级往往会引发硬分叉或者软分叉,并带来负面的影响。
以比特币为例,扩容升级会导致部分节点以最新的区块容量运行,而另一部分维持原状,这就产生了硬分叉;
更进一步,对于以太坊The DAO问题引发的硬分叉(被迫升级),产生了ETH和ETC重资产和社区分裂的副作用。

区块链系统升级导致交易所停止充提会增加交易所的维护成本,造成直接的经济损失。
区块链系统升级意味着需要提前通知交易所停止充提,交易所需要获取最新的区块链系统代码,编译测试并部署系统;
在此过程中如果系统升级不顺利,需要联系系统开发人员,沟通协调以完成升级。
更进一步,即使交易所顺利完成升级,造成的经济损失仍无法避免。
以比特币为例,近一年日均交易量为x美元~\cite{coinmarketcap},假设日充提比例占日交易量的x\%,充提停止一小时对交易所造成的损失为x美元。

传统中心化系统升级方式分为离线升级和在线升级~\cite{}。
离线升级方式是由软件公司发布离线更新包(安装程序),客户端下载离线更新包并执行安装程序,更新系统库文件、修改系统配置;
在线升级方式是通过在线检查本地系统版本与服务器系统版本差异,在线下载更新安装程序,替换本地待更新的文件。

传统分布式系统(服务器集群、内容分发网络、点对点系统、传感网络)升级~\cite{ajmani2003scheduling},由可信机构发布安装更新到相对中心化的升级节点,升级节点更新系统版本号,并发布升级系统的下载服务;
当分布式节点检查到新版本时,分布式节点的更新层下载更新包,交由节点内部的更新管理模块完成更新升级,此过程可以做到不中断对外服务的升级。

综上,传统中心化系统与传统分布式系统升级引入了中心化或相对中心化的机制,这不符合区块链系统去中心化、节点不可信的特性,无法应用在区块链系统升级上。

Tezos~\cite{tezoswhitepaper}是首次提出进行自我修正升级的公链,修正提案由StakeHolders投票选出,通过set\_test\_protocol以及promote\_test\_protocol两个过程完成协议替换,并在此之后生效。
由于Tezos系统的实现语言天然支持动态编译,自我修正升级过程无需分叉。

然而,Tezos的升级方式存在局限性。
Tezos升级是对原有的协议进行替换,在完成替换之后,想要运行原有的协议变得不可能,需要再进行一次协议替换过程来完成升级的回滚,无法在某个区块高度上同时运行多个版本的系统协议;
更进一步,当请求不同版本的协议时,无法并发地执行不同版本的协议,并正确地返回请求结果。

基于区块链系统的升级现状,本文提出了区块链运行时环境(BRE)。
区块链系统需要升级的部分(协议)以源代码的形式封装到交易的数据字段,并打包上链;
区块链运行时环境会解析这种特殊类型的交易,进而解析交易的数据字段得到协议源代码;
协议源代码会被编译生成协议表示,并持久化到数据库中,完成区块链系统的升级。
当客户端请求达到区块链系统时,区块链运行时环境通过版本管理模块获取对应的协议代码,交由执行引擎执行协议代码,返回客户端请求的结果。

区块链运行时环境解决了区块链系统升级的问题。
区块链系统升级只需发送一条特殊的交易(以多签的方式对该交易签名),整个过程不会产生硬分叉或者软分叉,交易所的充提服务无需停止。
除此之外,区块链运行时环境的执行引擎(即时编译)使协议多版本的执行得以实现,协议多版本的并发执行成为可能。
更进一步,执行引擎实现了软件事务内存,保证了在客户端的并发请求下,执行结果的正确性。

本文的其它部分按如下组织。
第二章阐述了当前区块链系统的升级现状、即时编译技术以及软件事务内存;
第三章概括了系统的抽象模型,系统需要满足的性质以及设计原则;
第四章描述了系统的整体架构,系统的构成以及各个模块的功能与交互;
第五章详细说明了各模块的实现细节,包括设计思想和算法实现;
第六章通过实验评估了每个模块的性能以及系统整体的性能,并通过实验分析,解释说明系统的先进性;
第七章对本文总结概括。
